\documentclass[UTF8, 12pt]{ctexart}
\usepackage{xcolor}
\xeCJKsetup{CJKmath=true} 
% 使用颜色
\definecolor{orange}{RGB}{255,127,0} 
\definecolor{violet}{RGB}{192,0,255} 
\definecolor{aqua}{RGB}{0,255,255} 
\usepackage{geometry}
\setcounter{tocdepth}{5}
\setcounter{secnumdepth}{5}
% 设置四级目录与标题
\geometry{papersize={21cm,29.7cm}}
% 默认大小为A4
\geometry{left=3.18cm,right=3.18cm,top=2.54cm,bottom=2.54cm}
% 默认页边距为1英尺与1.25英尺
\usepackage{indentfirst}
\setlength{\parindent}{2.45em}
% 首行缩进2个中文字符
\usepackage{amssymb}
% 因为所以与其他数学拓展
\usepackage{amsmath}
% 数学公式
\usepackage{setspace}
\renewcommand{\baselinestretch}{1.5}
% 1.5倍行距
\usepackage{pifont}
% 圆圈序号
\usepackage{tikz}
% 绘图
\usepackage{array}
% 设置表格行距
\usepackage[colorlinks,linkcolor=black,urlcolor=blue]{hyperref}
% 超链接
\author{guohuiyuan}
\title{罗尔定理}
\date{}
\begin{document}
\renewcommand{\arraystretch}{1.5}
% 表格高1.5倍
\maketitle
\pagestyle{empty}
\thispagestyle{empty}
\tableofcontents
\thispagestyle{empty}
\newpage

\section{问题} % 创建章节标题
设$f(x)$在$[a,b]$上连续,在$(a,b)$内可导,$0<a<b$,且$f(a)=f(b)=0$,证明:$\exists \xi \in(a,b), f^{\prime}(\xi)+p(\xi)f(\xi)=0$

\section{示例1}
设$f(x)$在$[a,b]$上连续,在$(a,b)$内可导,$0<a<b$,且$f(a)=f(b)=0$,证明:
\begin{enumerate}
\item 至少存在一点$\xi \in(a, b)$, 使得$2 f(\xi)+\xi f^{\prime}(\xi)=0$;
\item 至少存在一点$\eta \in(a, b)$, 使得$2 \eta f(\eta)-f^{\prime}(\eta)=0$.
\end{enumerate}

解答:
\begin{enumerate}
\item 
$f^{\prime}(\xi)+ \tfrac{2}{\xi}f(\xi)=0$

$\because y^*=e^\lambda f(x) $

${y^*}^{\prime}=e^\lambda(f^{\prime}(x)+\lambda^{\prime}f(x))$

$\therefore$ 令$\lambda=\int \tfrac{2}{x}  \,dx=2\ln x$

构造$g(x)=e^{2\ln x}f(x)=x^2f(x)$

${g(x)}^{\prime}=x^2(f^{\prime}(x)+f(x)\tfrac{2}{x})$

$\because g(a)=0,g(b)=0$

$\therefore$ 根据罗尔定理$,\xi \in(a, b),g^{\prime}(\xi)=0$,即$\xi^2(f^{\prime}(\xi)+f(\xi)\tfrac{2}{\xi})=0$

$\therefore$ 至少存在一点$\xi \in(a, b), $使得$2 f(\xi)+\xi f^{\prime}(\xi)=0$

\item
同理构造$g(x)=e^{-x^2}f(x)$
\end{enumerate}

\section{示例2}
$设  f(x)  在  [0,1]  上二阶可导, 且  \lim _{x \rightarrow 0^{+}} \frac{f(x)}{x}=\lim _{x \rightarrow 1^{-}} \frac{f(x)}{x-1}=1 , 证明:$
$(I) 至少存在一点  \xi \in(0,1) , 使得  f(\xi)=0 ;$
$(II) 至少存在一点  \eta \in(0,1) , 使得  f^{\prime \prime}(\eta)=f(\eta) .$

解答:

\section{结论}
遇见$f^{\prime}(\xi)+p(\xi)f(\xi)=0$类似格式的题目, 构造$g(x)=e^{\int p(x) \,dx } f(x)$, 然后使用罗尔定理解题

\end{document} % 结束文档