\documentclass[UTF8, 12pt]{ctexart}
% UTF8编码,ctexart现实中文
\usepackage{color}
% 使用颜色
\usepackage{geometry}
\setcounter{tocdepth}{4}
\setcounter{secnumdepth}{4}
% 设置四级目录与标题
\geometry{papersize={21cm,29.7cm}}
% 默认大小为A4
\geometry{left=3.18cm,right=3.18cm,top=2.54cm,bottom=2.54cm}
% 默认页边距为1英尺与1.25英尺
\usepackage{indentfirst}
\setlength{\parindent}{2.45em}
% 首行缩进2个中文字符
\usepackage{setspace}
\renewcommand{\baselinestretch}{1.5}
% 1.5倍行距
\usepackage{amssymb}
% 因为所以
\usepackage{amsmath}
% 数学公式
\usepackage[colorlinks,linkcolor=black,urlcolor=blue]{hyperref}
% 超链接
\begin{document} % 开始文档


\title{罗尔定理} % 标题
\author{guohuiyuan} % 作者
\date{\today} % 日期

\maketitle % 创建标题

\section{问题} % 创建章节标题
设$f(x)$在$[a,b]$上连续,在$(a,b)$内可导,$0<a<b$,且$f(a)=f(b)=0$,证明:$\exists \xi \in(a,b), f^{\prime}(\xi)+p(\xi)f(\xi)=0$

\section{示例}
设$f(x)$在$[a,b]$上连续,在$(a,b)$内可导,$0<a<b$,且$f(a)=f(b)=0$,证明:
\begin{enumerate}
\item 至少存在一点$\xi \in(a, b)$, 使得$2 f(\xi)+\xi f^{\prime}(\xi)=0$;
\item 至少存在一点$\eta \in(a, b)$, 使得$2 \eta f(\eta)-f^{\prime}(\eta)=0$.
\end{enumerate}

解答:
\begin{enumerate}
\item 
$f^{\prime}(\xi)+ \tfrac{2}{\xi}f(\xi)=0$

$\because y^*=e^\lambda f(x) $

${y^*}^{\prime}=e^\lambda(f^{\prime}(x)+\lambda^{\prime}f(x))$

$\therefore$ 令$\lambda=\int \tfrac{2}{x}  \,dx=2\ln x$

构造$g(x)=e^{2\ln x}f(x)=x^2f(x)$

${g(x)}^{\prime}=x^2(f^{\prime}(x)+f(x)\tfrac{2}{x})$

$\because g(a)=0,g(b)=0$

$\therefore$ 根据罗尔定理$,\xi \in(a, b),g^{\prime}(\xi)=0$,即$\xi^2(f^{\prime}(\xi)+f(\xi)\tfrac{2}{\xi})=0$

$\therefore$ 至少存在一点$\xi \in(a, b), $使得$2 f(\xi)+\xi f^{\prime}(\xi)=0$

\item
同理构造$g(x)=e^{-x^2}f(x)$
\end{enumerate}

\section{结论}
遇见$f^{\prime}(\xi)+p(\xi)f(\xi)=0$类似格式的题目, 构造$g(x)=e^{\int p(x) \,dx } f(x)$, 然后使用罗尔定理解题

\end{document} % 结束文档